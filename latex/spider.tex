\documentclass[12pt,a4paper, numbers=endperiod, parindent=half, twoside]{scrartcl}
\setlength{\parskip}{0.5em}
\setlength{\parindent}{1em}
%\usepackage[applemac]{inputenc}
\usepackage[ngerman]{babel}
\usepackage[scaled=0.92]{helvet}
\renewcommand\familydefault{phv}
\usepackage[T1]{fontenc}
\usepackage{setspace} 
\onehalfspacing 
%%%%%%%%%%%%%%%%%%%%%%%%%%%%%%
%% ToDo-Notes %%
%%%%%%%%%%%%%%%%%%%%%%%%%%%%%%
\usepackage[ngerman]{todonotes}
% \usepackage[T1]{fontenc}
\usepackage[utf8]{inputenc}

%%Packages for Radar graph
\usepackage[upright]{fourier} 
%\usepackage[usenames,dvipsnames]{xcolor}
\usepackage{tikz}
\usepackage{pgf-umlsd}
\usepgflibrary{arrows} % for pgf-umlsd
\usepackage{tkz-kiviat,numprint,fullpage} 
%\usetikzlibrary{arrows}
\usetikzlibrary{arrows,shadows} % for pgf-umlsd
%\usepackage[underline=true,rounded corners=false]{pgf-umlsd}

\usepackage{eurosym}

\usepackage{booktabs}

\usepackage{enumitem}

\usepackage{geometry}
\geometry{a4paper,left=25mm,right=20mm,top=27mm,bottom=1.5 cm,headheight=25mm, footskip=1.5cm, includeheadfoot}
\setkomafont{pagefoot}{\normalfont\sffamily}

% create a shortcut to typeset table headings
\usepackage{lscape}
\newcommand\tabhead[1]{\small\textbf{#1}}
\newcommand{\tabitem}{~~\llap{\textbullet}~~}

\usepackage{numprint}
\usepackage{tabularx}
% Tabellenbreite automatisch berechnen mit tabularX
\usepackage{booktabs,multirow}
\usepackage{longtable}
\usepackage{forloop,supertabular}
\usepackage[autostyle=true,german=quotes]{csquotes}
\usepackage{graphicx}
\usepackage{wrapfig}

%\usepackage{subfig}
\usepackage{caption}
\usepackage{subcaption}
\usepackage{placeins}
\graphicspath{{img/}{images/}{./}}
\DeclareGraphicsExtensions{.pdf,.png,.jpg}

\usepackage{graphicx}
\usepackage{scrpage2}

\usepackage{tikz}
\usetikzlibrary{matrix}
\usetikzlibrary{shapes,arrows}
\colorlet{helpful}{lime!70}
\colorlet{harmful}{red!30}
\colorlet{internal}{yellow!20}
\colorlet{external}{cyan!30}
\colorlet{S}{helpful!50!internal}
\colorlet{W}{harmful!50!internal}
\colorlet{O}{helpful!50!external}
\colorlet{T}{harmful!50!external}

\newcommand{\texta}{Stärken\\ \tiny (zur Erreichung des Ziels)\par}
\newcommand{\textb}{Schwächen\\ \tiny (zur Erreichung des Ziels)\par}
\newcommand{\textcn}{Chancen\\ \tiny (Produkt\slash Firma und Team)\par}
\newcommand{\textdn}{Risiken\\ \tiny (Umgebung\slash Markteinflüsse)\par}

\newcommand{\back}[1]{\fontsize{60}{70}\selectfont #1}

% Gantt-Diagramm
\usepackage[]{pgfgantt}

\usepackage{url}
\urlstyle{sf} 
\usepackage[breaklinks=true]{hyperref}

\usepackage{mathtools}

\usepackage{tabularx}
\usepackage{booktabs}

% Fussnoten
\usepackage{savefnmark}

\usepackage[numbers]{natbib}
\bibpunct{[}{]}{;}{n}{,}{,}

%\hyphenation{Be-nutzer-ober-fläche}


\usepackage{listings}
\definecolor{lstBackground}{RGB}{245,245,245}
\lstset{
	language=Java,
	backgroundcolor=\color{lstBackground},
	basicstyle=\scriptsize,
	mathescape=true,
	captionpos=b,
	frame=tlbr,
	framesep=5pt}

\newcommand{\img}[4]{
	% #1 - source
	% #2 - label
	% #3 - caption
	% #4 width, relative to \textwidth
	\begin{figure}[!htb]
		\centering
		\includegraphics[width=#4\textwidth]{#1}
		\caption{#3}
		\label{#2}
	\end{figure}
}



\newcommand{\imgs}[7]{
	\begin{figure}[!htb]
		\begin{minipage}[b]{#7\textwidth}
			\centering
			\includegraphics[width=\textwidth]{#1}
			\caption{#3}
			\label{#2}
		\end{minipage}
		\hfill
		\begin{minipage}[b]{#7\textwidth}
			\centering
			\includegraphics[width=\textwidth]{#4}
			\caption{#6}
			\label{#5}
		\end{minipage}
	\end{figure}
}

\newcommand{\imgsheight}[8]{
	\begin{figure}[!htb]
		\begin{minipage}[b][#8]{#7\textwidth}
			\centering
			\includegraphics[width=\textwidth]{#1}
			\caption{#3}
			\label{#2}
		\end{minipage}
		\hfill
		\begin{minipage}[b][#8]{#7\textwidth}
			\centering
			\includegraphics[width=\textwidth]{#4}
			\caption{#6}
			\label{#5}
		\end{minipage}
	\end{figure}
}

\hyphenation{Privalino}
\hyphenation{Nicolai}


\presetkeys%
    {todonotes}%
    {inline}{}

\newenvironment{q}{\begin{quote}\it}{\end{quote}}
\newenvironment{qq}{\begin{quotation}\it}{\end{quotation}}
\setlength{\parskip}{1.5mm}
% Verhindern von "Schusterjungen" und "Hurenkindern"
 \clubpenalty = 10000
 \widowpenalty = 10000
 \displaywidowpenalty = 10000
 \tolerance=500 %Zeilenumbruch

\title{2016 Presidential Debates}
\date{\today}
%\author{Dr.-Ing.\,Nicolai Erbs \and Patrick Schneider \and Dr.-Ing.\,Quan Nguyen \and Leon van Dijk \and Dominik Püllen}
\author{Dr.-Ing.\,Nicolai Erbs}
% \usepackage[scaled]{uarial}
% \usepackage{fontspec}
% \setmainfont{Arial}
\setcounter{tocdepth}{2} 
\setcounter{secnumdepth}{3}

\begin{document}
\selectlanguage{ngerman}
\maketitle


% \clearpage% oder \cleardoublepage bei twoside
% \begingroup
% \renewcommand*{\chapterpagestyle}{empty}
% \pagestyle{empty}
% \tableofcontents
% \clearpage
% \endgroup

\renewcommand\todo[1]{}




% **********************************************************************************************
% NEW SECTION
% **********************************************************************************************




\begin{figure}[!hbt]
\centering
\begin{tikzpicture}
\tkzKiviatDiagram[label distance=4cm,
 radial = 20,
 gap = 0.80, 
 lattice = 10]{Flesch-Kincaid, Gunning fog, SMOG, Coleman-Liau, ARI}

%Politician   		tokens     c/w     w/s  lexdiv  	flesch    smog     	ari     	lix 		col_lia 		kincaid     fog  
%Cameron      	28960    4.16   18.92    9.89   	70.07   10.06    	7.98    	5.00   	10.70    	7.57   		10.62 
%\tkzKiviatLine[ultra thick,color=red, fill=black!20,opacity=.5](7.57, 10.62, 10.06, 10.70, 7.98)
 
\tkzKiviatLine[ultra thick,color=red, fill=black!20,opacity=.5](3.813,6.689,7.344,8.342,3.380) 
\tkzKiviatLine[thick, color=red, fill=black!20,opacity=.2](6.635,9.924,9.473,9.269,6.704) 

%Politician   		tokens     c/w     w/s  lexdiv  	flesch    smog     	ari     	lix 		col_lia 		kincaid     fog  
%Johnson       		12300    4.20   29.29    8.91   	59.81   12.27   	12.62   8.00   	10.53   	11.37   		14.81 
%\tkzKiviatLine[ultra thick,color=blue,mark=none, fill=blue!20,opacity=.5](11.37, 14.81, 12.27, 10.53, 12.62)
\tkzKiviatLine[ultra thick,color=blue,mark=none, fill=blue!20,opacity=.5](6.186,9.423,9.127,9.123,6.233)
\tkzKiviatLine[thick, color=blue,mark=none, fill=blue!20,opacity=.2](6.188,9.305,9.249,10.122,6.248)

%Politician   		tokens     c/w     w/s  lexdiv  	flesch    smog     	ari     	lix 		col_lia 		kincaid     fog  
%Shakespeare   	29866    3.43   19.26    9.50   	90.72    6.65    	4.70    	0.00    	7.40    		4.37    		7.23 
%\tkzKiviatLine[thin,mark=ball, mark=none,color =black](4.37, 7.23, 6.65, 7.40, 4.70) 
\tkzKiviatLine[thin,mark=ball, mark=none,color =black](2.175, 3.615, 3.325, 3.70, 2.35) 

%Politician   		tokens     c/w     w/s  lexdiv  	flesch    smog     	ari     	lix 		col_lia 		kincaid     fog  
%Utopia'       		49108    3.94   65.30   13.78   36.97   14.10   	27.63   99.00   9.53   		22.42   	25.90 
%\tkzKiviatLine[thin,mark=ball, mark=none, color =black](22.42, 25.9, 14.1, 9.53, 27.63) 

\tkzKiviatGrad[unity=2](0)
%\tkzKiviatGrad[](0)

\node[draw=none, color=red, font=\Large] at (-1.5,3.7) {Trump};
\node[draw=none, color=blue, font=\Large] at (4.7,3.7) {Clinton};
\node[draw=none, color=red, font=\small] at (3.2,-5) {Pence};
\node[draw=none, color=blue, font=\small] at (-1.5,-5) {Kaine};
\end{tikzpicture}

\caption{Profil der geschriebenen Texte von Kindern und Erwachsenen auf Basis ausgewählter Merkmale.}
\label{fig:radarFeatures}
\end{figure}


\end{document}